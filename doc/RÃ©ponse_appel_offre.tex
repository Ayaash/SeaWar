\documentclass{article}

\usepackage[utf8]{inputenc}
\usepackage[T1]{fontenc}
\usepackage[francais]{babel}

\title{Projet Sea-War : réponse à l'appel d'offre}
\author{Hugo \bsc{Ayats}, Laurent \bsc{Pichollet},\\ Thibaut \bsc{Lausecker}, Valentin \bsc{Louveau},\\ Amine \bsc{Abdellaoui}, Friedrich \bsc{Gonda}}
\date{\today}

\begin{document}

\maketitle

\section*{Présentation de l'équipe}

Notre équipe, composé de six ingénieurs fraîchement diplômés, a été interpellé par votre appel d'offre et souhaiterait y répondre. En effet, nos compétences en développement logiciel, notamment en Java, nous permettent d'avoir confiance en nos capacités. De plus, ce projet nous attire, étant nous-même impliqués dans le domaine du jeu vidéo et de l'\textit{e-sport}. Nous nous permettons donc de vous contacter en vue de réaliser ce projet.



\section*{Proposition de solution}

Nous prévoyons de développer cette application en Java, en utilisant la bibliothèque \textit{Slick2D}, bibliothèque connue pour ses fonctionnalités utiles au développement de jeux vidéos 2D. Le projet sera principalement basé sur un patron observeur/observé, ce qui permettra d'avoir un code organisé de la manière la plus simple entre l'interface et le modèle du jeu. 

Nous souhaiterions fonctionner selon une méthodes de développement continu, ce qui aura comme avantage de permettre au client un accès direct et permanant à l'avancement du projet. De plus, nous prévoyons d'effectuer deux livraisons intermédiaire au cours du développement, dont les dates, encore non spécifiés, seront à déterminer avec le client.

Notre équipe, formé aux méthodes agiles, souhaiterait appliquer ces méthodes à ce projet, entre autre grâce à une organisation non hiérarchique de l'équipe, de la programmation piloté par les tests et éventuellement du développement en binôme.

\section*{Conclusion}

Comme dit ci-dessus, notre équipe est donc à la fois motivée, dynamique et compétente pour la réalisation du projet. Ainsi, nous souhaiterions pouvoir vous rencontrer physiquement afin de discuter de cela plus longuement.

\end{document}
